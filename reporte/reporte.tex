\documentclass[spanish,12pt,letterpaper]{article}
\usepackage[margin=2cm]{geometry}
\usepackage[T1]{fontenc}
\usepackage{graphicx}
\usepackage{mathtools}
\usepackage{amssymb}
\usepackage{amsthm}
\usepackage{thmtools}
\usepackage{nameref}
\usepackage{babel}
\usepackage{mathrsfs}
\usepackage{hyperref}
\usepackage[backend=biber,style=numeric]{biblatex}
\usepackage{algorithm}
\usepackage[noend]{algpseudocode}

\title{\textbf{Subset Sum Problem}}
\author{Victor Rosales Jaimes}
\date{}
\begin{document}
	\maketitle
	\section{Introducción}
	
	Subset Sum Problem es un desafío fundamental en informática que implica encontrar un subconjunto de un conjunto dado cuya suma sea igual a un valor específico. Este problema es conocido por ser NP-duro, lo que significa que es computacionalmente intratable para un gran número de elementos. Se ha utilizado en una variedad de campos, como criptografía, teoría de la complejidad y optimización combinatoria.\\
	
	
	La Búsqueda Tabú, por otro lado, es una metaheurística de optimización matemática que guía un procedimiento de búsqueda local para explorar el espacio de soluciones más allá del óptimo local. Este enfoque ha demostrado ser efectivo en la resolución de problemas difíciles, como el Problema del Agente Viajero (TSP), que es otro desafío computacionalmente complejo. La Búsqueda Tabú se ha aplicado con éxito en la resolución de problemas de enrutamiento y optimización, lo que destaca su versatilidad y eficacia en diversas aplicaciones prácticas. Estos dos problemas, el Subset Sum y la Búsqueda Tabú, representan desafíos significativos en informática y han sido objeto de extensa investigación y aplicación en una variedad de campos.
	
	\section{Subset Sum Problem}
	
	Subset Sum problem es una variante de Knapsack problem que viene dado como sigue: dada un conjunto de $n$ elementos y una mochila, con
	
	$$
	\begin{aligned}
		w_{j} & =\text{ peso del elemento } j \\
		c & =\text{ capacidad de la mochila}
	\end{aligned}
	$$
	
	seleccionar un subconjunto de elementos cuyo peso total sea lo más cercano posible, sin exceder, $c$, es decir,
	
	
	\begin{align}
		\operatorname{maximizar} z= & \sum_{j=1}^{n} w_{j} x_{j} \\
		\text{sujeta a} & \sum_{j=1}^{n} w_{j} x_{j} \leq c, \hspace{1cm} x_{j}=0 \text{ o } 1, \quad j \in N=\{1, \ldots, n\}.
	\end{align}
	
	
	donde
	
	$$
	x_{j}= \begin{cases}1 & \text{ si el elemento } j \text{ está seleccionado } \\ 0 & \text{ de lo contrario }\end{cases}
	$$
	
	El problema está relacionado con la ecuación diofántina:
	

	\begin{align}
		& \sum_{j=1}^{n} w_{j} x_{j}=\hat{c} \hspace{1cm} x_{j}=0 \text{ o } 1, \quad j=1, \ldots, n,
	\end{align}
	
	en el sentido de que el valor óptimo del Subset Sum Problem es el $\hat{c}$ más grande tal que las fórmulas anteriores tiene una solución.\\
	
	En el contexto de nuestro proyecto, consideraremos elementos enteros, los cuales pueden ser tanto positivos como no positivos. A diferencia de la convención común de utilizar únicamente números enteros positivos, aquí se permitirá cualquier número entero, inclusive valores no positivos, para la capacidad de la mochila.
	
	\subsection{Función de costo}
	
	La fórmula (3) presenta una función de costo diseñada originalmente para el Problema de la Suma de Subconjuntos (SSP) con números no negativos. En la adaptación a números enteros en este proyecto, denominaremos $c$ como la capacidad de la mochila, a la que referiremos como el \textit{target}. Por lo tanto, definimos:
	
	\begin{align*}
		U &: \text{Conjunto de enteros o elementos iniciales} \\
		S &: \text{Subconjunto de } U\\
		c &: \text{\textit{Target}}
	\end{align*} 
	
	La función de costo $f$ se define como:
	
	\[
	f(S) = \left|\sum_{s\in S} s - c\right|  = \left|c - \sum_{s\in S} s \right|
	\]
	
	El objetivo consiste en minimizar la función de costo, idealmente alcanzando un valor de $0$.
	
	\subsection{Soluciones}
	
	Consideramos cualquier subconjunto como una solución al problema. Sin embargo, no todos los subconjuntos se acercan a $0$, y mucho menos son iguales a $0$. Clasificamos las soluciones de la siguiente manera:
	
	\begin{itemize}
		\item \textbf{Soluciones factibles}: Son aquellos subconjuntos $S$ para los cuales la evaluación de su función de costo da como resultado $0$.
		\item \textbf{Soluciones no factibles}: Son aquellos subconjuntos $S$ para los cuales la evaluación de su función de costo da como resultado un valor diferente de $0$.
	\end{itemize}
	
	Es importante destacar que el tamaño del conjunto no es relevante, ya que no buscamos maximizar ni minimizar esta variable, aunque esto podría considerarse en casos particulares.
	\subsection{Solución vecina}
	La función para generar un vecino para el Subset Sum Problem mediante operaciones aleatorias: adición, eliminación o intercambio de elementos en una solución actual. Se utiliza un valor aleatorio para decidir la operación a realizar. Si es menor que 0.9, se elige entre agregar o eliminar un elemento según la suma actual. Si es mayor o igual a 0.9, se intercambian dos elementos distintos. La función recalcula el costo y registra la última operación y los índices modificados para evaluar y mejorar la solución actual.
	\begin{algorithm}
		\caption{Generación de Vecino para Subset Sum}
		\begin{algorithmic}[1]
			\Procedure{GenerarVecino}{}
			\State Genera un valor aleatorio $randomValue$ en el rango [0, 1).
			\If{$randomValue < 0.9$}
			\State \textbf{AplicarOperacionAdicion()} \Comment{Ver función en detalles}
			\Else
			\State \textbf{AplicarOperacionIntercambio()} \Comment{Ver función en detalles}
			\EndIf
			\EndProcedure
		\end{algorithmic}
	\end{algorithm}
	
	\begin{algorithm}
		\caption{AplicarOperacionAdicion}
		\begin{algorithmic}[1]
			\Procedure{AplicarOperacionAdicion}{}
			\If{$\text{sum} < \text{target}$}
			\State Selecciona un índice aleatorio $indexToAdd$.
			\If{$\text{byteMap}[indexToAdd] = 0$}
			\State $\text{byteMap}[indexToAdd] \gets 1$
			\State Registra la última operación como adición (1).
			\State Llama a \textbf{CalcularCosto()}.
			\State \textbf{return}
			\EndIf
			\Else
			\State Selecciona un índice aleatorio $indexToRemove$.
			\If{$\text{byteMap}[indexToRemove] = 1$}
			\State $\text{byteMap}[indexToRemove] \gets 0$
			\State Registra la última operación como eliminación (2).
			\State Llama a \textbf{CalcularCosto()}.
			\State \textbf{return}
			\EndIf
			\EndIf
			\EndProcedure
		\end{algorithmic}
	\end{algorithm}
	
	\begin{algorithm}
		\caption{AplicarOperacionIntercambio}
		\begin{algorithmic}[1]
			\Procedure{AplicarOperacionIntercambio}{}
			\State Selecciona dos índices aleatorios distintos $index1$ y $index2$.
			\If{$\text{byteMap}[index1] \neq \text{byteMap}[index2]$}
			\State Intercambia los bits en las posiciones $index1$ e $index2$ de $\text{byteMap}$.
			\State Registra la última operación como intercambio (3).
			\State Llama a \textbf{CalcularCosto()}.
			\EndIf
			\EndProcedure
		\end{algorithmic}
	\end{algorithm}
	
	
	Es importante destacar que, independientemente de la operación realizada, la diferencia en tamaño entre una solución y su vecina es de, como máximo, 1. Esto nos habilita para llevar a cabo una búsqueda local de manera efectiva.
	
	\section{Búsqueda Tabú}
	
	La Búsqueda Tabú (Tabu Search - TS) es una metaheurística que destaca por su utilización de memoria adaptativa y estrategias específicas para la resolución de problemas. Su enfoque se basa en la explotación de estrategias inteligentes apoyadas en procesos de aprendizaje. La memoria adaptativa de TS se centra en cuatro dimensiones clave: recencia, frecuencia, calidad e influencia, referenciando la historia del proceso de resolución.\\
	
	Con raíces en métodos diseñados para superar barreras de factibilidad u optimalidad local, TS ha evolucionado para imponer y eliminar cotas de manera sistemática, permitiendo la exploración de regiones previamente no consideradas. Su distintiva característica radica en el uso de memoria adaptativa y estrategias específicas para la resolución de problemas, dando origen al enfoque basado en memoria y estrategia en metaheurísticas.\\
	
	TS, pionero en el enfoque basado en memoria, ha destacado por su énfasis en el diseño estructurado para explotar patrones históricos de búsqueda, a diferencia de métodos que dependen casi exclusivamente de la aleatorización. Los principios fundamentales de la Búsqueda Tabú fueron desarrollados en la década de 1980 y principios de la década de 1990, consolidándose en el libro "Tabu Search". Su éxito en la resolución de problemas de optimización desafiantes, especialmente en aplicaciones del mundo real, ha impulsado numerosas aplicaciones TS en los últimos años.\\
	
	La filosofía de la Búsqueda Tabú se centra en derivar y explotar estrategias inteligentes basadas en procedimientos de aprendizaje, respaldadas por un marco de memoria adaptativa que no solo aprovecha la historia del proceso de resolución, sino que también requiere la creación de estructuras para facilitar esta explotación. Las estructuras de memoria de TS se apoyan en cuatro dimensiones principales: recencia, frecuencia, calidad e influencia, y se fundamentan en antecedentes de conectividad y estructuras lógicas para crear procesos eficaces de resolución de problemas.
	
	\subsection{Pseudocodigo}
	La Búsqueda Tabú, un algoritmo metaheurístico innovador, se caracteriza por su capacidad para explorar el espacio de búsqueda de soluciones y superar obstáculos locales, lo cual es ejemplificado en el pseudocódigo proporcionado. Inicia con la definición de una función de vecindario \(N\) y la elección de un algoritmo de búsqueda local (\textit{localSearch()}) para este vecindario específico. La variable \(t\) se inicializa a cero, marcando el comienzo de las iteraciones.\\
	
	La generación de una solución inicial, \(s_t\), constituye el punto de partida, a menudo lograda mediante una heurística de construcción. La mejor solución actual, \(s_{\text{best}}\), se establece como \(s_t\). La estructura de datos de memoria \(H_t\) se inicializa para registrar la historia del proceso de resolución, considerando la solución actual, la mejor hasta el momento y el tiempo de iteración actual.\\
	
	En cada iteración, determinada por criterios de terminación, se actualiza el tiempo \(t\). Se selecciona un subconjunto \(NH_t(s_t)\) del vecindario \(N(s_t)\), y se busca la mejor solución, \(s_t\), con respecto a alguna función objetivo \(f\) dentro de este subconjunto. Posteriormente, se actualiza la memoria tabú, \(H_t\), mediante una función \(F\) que incorpora información relevante de la iteración actual.\\
	
	La evaluación de la función objetivo (\(f(s_t)\)) permite determinar si la solución actual mejora la mejor encontrada hasta el momento (\(s_{\text{best}}\)). En caso afirmativo, se actualiza \(s_{\text{best}}\) con la solución actual. El proceso se repite hasta que se cumplen los criterios de terminación, permitiendo que la Búsqueda Tabú explore y adapte su memoria de manera dinámica para superar óptimos locales y converger hacia soluciones de alta calidad en el espacio de búsqueda.\\
	
	
	\begin{algorithm}
		\caption{Búsqueda Tabú}
		\begin{algorithmic}[1]
			\Procedure{BúsquedaTabú}{}
			\State Define una función de vecindario $N$ y selecciona un algoritmo de búsqueda local \textit{localSearch()} para $N$.
			\State $t \gets 0$
			\State $s_t \gets$ Genera una solución inicial
			\State $s_{\text{best}} \gets s_t$
			\State $H_t \gets (s_t, s_{\text{best}}, t)$ \Comment{Inicializa la estructura de datos de memoria}
			\While{no se cumplan los criterios de terminación($s_t, H_t$)}
			\State $t \gets t + 1$
			\State Genera un subconjunto $NH_t(s_t) \subseteq N(s_t)$
			\State $s_t \gets \text{Mejor } \{s | s \in NH_t(s_t)\} \text{ con respecto a } f \text{ o alguna } f$
			\State $H_t \gets F(f, s_t, H_t)$
			\If{$f(s_t) < f(s_{\text{best}})$}
			\State $s_{\text{best}} \gets s_t$
			\EndIf
			\State Actualiza la estructura de datos de memoria $H_t$
			\EndWhile
			\State \textbf{return} $s_{\text{best}}$
			\EndProcedure
		\end{algorithmic}
	\end{algorithm}
	
	\section{Sistema}
	\subsection{Instancia}
	En el contexto de este estudio, utilizamos un subconjunto de los números enteros para nuestro conjunto universo U. Estos están listados y separados por comas en un archivo. En el marco de la experimentación, empleamos dos conjuntos, cada uno de tamaño 1000.
	\subsection{Constantes del sistema}
	El sistema se ha implementado en Java, y en él se han implementado los dos algoritmos descritos en las secciones 2.3 y 3.1. A continuación, se enumeran las variables utilizadas en el sistema, que se han determinado de manera experimental y se ha notado que son las más adecuadas para este caso particular:
	
	\begin{itemize}
		\item Tamaño de lista tabu ($sizeTabuList$): $100000$
		\item Número de iteraciones ($iteraciones$): $1100000$
	\end{itemize}
	\subsection{Entrada}
	El sistema incorpora un componente aleatorio al buscar una solución vecina. Por esta razón, el sistema recibe como entrada un número entero que funciona como semilla para el generador de números aleatorios, lo que permite que los resultados sean reproducibles.
	
	Además, se requiere una lista de números que serán el conjunto universo. Este archivo esta compuesto por números separados por comas, el ultimo entero en el archivo es el target al que deseamos llegar. 
	
	En resumen, el sistema recibe los siguientes datos de entrada:
	\begin{itemize}
		\item Nombre del archivo que contiene una lista de números separados por comas.
		\item Una semilla aleatoria.
	\end{itemize}
	\section{Resultados}
	Como mencionamos en la sección anterior, tenemos dos instancias de tamaño 1000 las cuales cumplen con :
	\begin{enumerate}
		\item[Instancia 1] : compuesta por números enteros sin distinción  donde el target es $-466982$
		\item[Instancia 2]: compuesta por número enteros positivos donde el target es $4630441$
	\end{enumerate}
	
	Utilizando las constantes del sistema definidas en la sección 4.2, realizamos pruebas con $1000$ semillas en la primer instancia. Cada semilla produjo resultados distintos, y determinamos que el costo mínimo obtenido por el sistema se alcanza cuando el valor de la semilla es $57$, con un valor de la función de costo de $0$. El subconjunto encontrado es el siguiente: \\
	
	102588 ,247742 ,100334 ,-166875 ,218407 ,489544 ,-311915 ,50238 ,171106 ,-36773 ,127247 ,-478975 ,232072 ,143457 ,-454767 ,-306746 ,245269 ,383872 ,90099 ,-13442 ,398467 ,140645 ,196886 ,401104 ,423967 ,416067 ,391976 ,493985 ,-130183 ,-492995 ,245091 ,-35062 ,-134875 ,488743 ,-300981 ,136511 ,302707 ,-32000 ,-227802 ,347100 ,333783 ,327809 ,135637 ,459960 ,487063 ,-404280 ,407436 ,-55960 ,244250 ,484063 ,-231191 ,113012 ,470205 ,153037 ,331014 ,440686 ,47110 ,-307440 ,-421099 ,-197635 ,-71417 ,95398 ,278525 ,133833 ,-478680 ,-65083 ,-156146 ,334796 ,278904 ,231759 ,-272528 ,25913 ,-421766 ,40245 ,-359858 ,266343 ,-413216 ,-71261 ,6144 ,-373426 ,60091 ,16357 ,200011 ,418023 ,-461465 ,-190150 ,292757 ,-94458 ,-408760 ,-254264 ,33990 ,-27548 ,-319494 ,360814 ,260770 ,-164202 ,-427139 ,-121998 ,-96678 ,-180981 ,328204 ,211044 ,-252671 ,50872 ,144041 ,-219332 ,-199261 ,21525 ,45648 ,120367 ,221994 ,-479028 ,288594 ,279971 ,303032 ,6591 ,-132880 ,252718 ,-362226 ,142795 ,-60157 ,-358217 ,-272011 ,337506 ,-339950 ,-333112 ,132022 ,-10994 ,73409 ,273853 ,153255 ,102164 ,-227499 ,445876 ,2761 ,-231672 ,-289062 ,-115840 ,-19656 ,54540 ,-386350 ,-270487 ,-206311 ,270901 ,-209626 ,-130003 ,78749 ,204465 ,-454088 ,208188 ,480884 ,207575 ,380923 ,158042 ,-358133 ,-420023 ,340597 ,-50660 ,287772 ,55270 ,91360 ,-459307 ,157807 ,92465 ,462053 ,83648 ,259622 ,302013 ,-335803 ,-275280 ,-488992 ,-161520 ,-99481 ,-498341 ,419719 ,92865 ,484901 ,180023 ,-258505 ,269252 ,248036 ,289176 ,418255 ,105233 ,198621 ,-416920 ,-447761 ,10625 ,-320429 ,-88299 ,392458 ,366587 ,-342903 ,190573 ,-42557 ,-169017 ,475560 ,-135068 ,-163412 ,356536 ,3826 ,-121200 ,236590 ,-251447 ,-142455 ,472755 ,-120250 ,-265638 ,133166 ,307222 ,-465136 ,361707 ,-52253 ,-491214 ,-319686 ,373330 ,26643 ,-375386 ,-292493 ,-386672 ,-263832 ,-451682 ,-315828 ,335067 ,-289783 ,224544 ,395634 ,147332 ,-61174 ,-365557 ,-144961 ,365918 ,273103 ,428503 ,-464607 ,405520 ,392376 ,387616 ,255259 ,-208939 ,-148029 ,-317126 ,-471106 ,367233 ,-299488 ,29779 ,-124803 ,259963 ,-39975 ,-201829 ,-226888 ,365054 ,87216 ,-77038 ,-435483 ,176619 ,349895 ,347379 ,386171 ,326478 ,-262268 ,9029 ,498143 ,101547 ,-288440 ,-485045 ,51663 ,195856 ,201726 ,491909 ,31487 ,-222465 ,-172669 ,-243206 ,103178 ,-47633 ,-417962 ,-176545 ,-290313 ,-282758 ,215032 ,-424851 ,-278004 ,113993 ,-208138 ,-236977 ,-384659 ,94697 ,-436237 ,40834 ,356272 ,309183 ,-263234 ,-306822 ,-165371 ,382642 ,-492883 ,222859 ,-150266 ,311006 ,217877 ,237044 ,79391 ,29552 ,342875 ,456182 ,-153685 ,144019 ,483218 ,-236512 ,-236659 ,298055 ,424737 ,-371734 ,105581 ,329510 ,-232832 ,160250 ,77702 ,462146 ,-368618 ,-318408 ,45507 ,15879 ,-211457 ,382207 ,-207676 ,-145751 ,-55613 ,-360867 ,13158 ,-149730 ,-294063 ,324735 ,102412 ,-110230 ,-401084 ,-322893 ,-36467 ,-343586 ,298955 ,475518 ,-122889 ,-115711 ,347287 ,28495 ,315286 ,-270996 ,-450675 ,422928 ,-83311 ,-353227 ,198223 ,-377541 ,-310236 ,-219752 ,-365920 ,102272 ,-137761 ,203280 ,470582 ,150697 ,-452264 ,165698 ,-357023 ,-189194 ,-171934 ,-58554 ,-351179 ,135326 ,-111217 ,389314 ,-159821 ,-3925 ,-112790 ,-156729 ,-369496 ,84088 ,-268618 ,126966 ,-413486 ,74910 ,-415938 ,-37467 ,-133764 ,-236256 ,-294372 ,-69134 ,-387468 ,422836 ,-356693 ,381643 ,-8086 ,-361187 ,-71738 ,-224759 ,-201975 ,22239 ,-313169 ,-429000 ,-229611 ,-475536 ,-442988 ,451912 ,-29252 ,-368540 ,-445896 ,139819 ,-351380 ,-125089 ,302867 ,215747 ,339238 ,108787 ,-488858 ,-357823 ,179431 ,197897 ,-358764 ,-110539 ,425638 ,-442470 ,165364 ,-427023 ,-268328 ,-282222 ,77444 ,239962 ,-5133 ,230516 ,434587 ,204729 ,-170666 ,433739 ,-4976 ,-261091 ,126485 ,323723 ,-140704 ,233295 ,325046 ,205844 ,19186 ,232113 ,-233820 ,-360886 ,-317133 ,-425906 ,312383 ,39110 ,55604 ,286637 ,339965 ,-309328 ,418908 ,-188686 ,-360446 ,275243 ,146141 ,168558 ,442427 ,460610 ,442639 ,-244927 ,-384248 ,337482 ,150787 ,-47078 ,-57059 ,24814 ,-273469 ,377917 ,-462601 ,308835 ,498601 ,-343952 ,-482163 ,-37535 ,-274983 ,376952 ,131702 ,454706 ,-111837 ,36410 ,491543 ,-172376 ,-113407 ,-72126 ,-101779 ,354445 ,-327475 ,-164651 ,-487994 ,-434002 ,-237547 ,-82529 ,-141311 ,388585 ,-429627 ,89499 ,114065 ,162458 ,-301333 ,58002 ,25826 ,370194 ,267968 ,142748 ,5371 ,-298572 ,368496 ,-484390 ,-282246 ,238729 ,-75006 ,158895 ,-157205 ,190500 ,121092 ,296759 ,12264 ,471757 ,-213658 ,-489736 ,425231 ,-136116 ,47139 ,260179 ,-145165 ,361818 ,-401484 ,-311631 ,368008 ,-48390 ,-444621 ,118142 ,31447 ,-1981 ,-133949 ,-382296 ,154385 ,430307 ,63426 ,459613 ,-388894 ,143031 ,253668 ,475032 ,264335 ,213381 ,-146327 ,190678 ,-371904 ,490078 ,4657 ,321900 ,17381 ,-170623 ,38393 ,278239 ,-305180 ,-208592 ,-129598 ,198769 ,-213751 ,-370267 ,-435315 ,155714 ,363283 ,196103 ,-168346 ,316812 ,325615 ,-494648 ,58890 ,-409782 ,-286010 ,428732 ,-227540 ,-386611 ,-312131 ,175196 ,-385404 ,38325 ,-339094 ,120449 ,396871 ,-323181 ,111916 ,179438 ,-364382 ,-179273 ,-448930 ,297782 ,272642 ,-324408 ,473214 ,178689 ,-135046 ,68886 ,39935 ,-367812 ,315279 ,-107811 ,-132391 ,190602 ,313600 ,-317560 ,-103579 ,388415 ,-399551 ,265925 ,-211319 ,212067 ,-188690 ,222787 ,-439921 ,290212 ,354535 ,301763 ,-111509 ,-428704 ,304420 ,-447529\\
	
	
	En el caso de la segunda instancia, probamos $100$ semillas y determinamos que el costo mínimo se alcanza cuando el valor de la semilla es $4$, con un costo de $0$. El subconjunto encontrado es el siguiente:\\
	
	97287 ,378352 ,21270 ,112695 ,620416 ,572646 ,55276 ,76330 ,56915 ,13625 ,908362 ,421996 ,121756 ,531348 ,129454 ,339788 ,172925
	
	\newpage
	\begin{thebibliography}{99}
		\bibitem{1} Martello, S., \& Toth, P. (1990). \textit{Knapsack problems: Algorithms and Computer Implementations}.
		
		\bibitem{2} Glover, F. and Laguna, M. (1997). \textit{Tabu Search, Kluwer Academic Publishers}
		
		\bibitem{3} Glover, F., \& Melián
		 B. (2003). Tabu Search. Inteligencia Artificial, Revista Iberoamericana de Inteligencia Artificial, 19, 29-48
	\end{thebibliography}
	
\end{document} 
